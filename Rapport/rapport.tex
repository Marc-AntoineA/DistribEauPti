\documentclass{article}
\usepackage[francais]{babel}
\usepackage[utf8]{inputenc}
\usepackage[T1]{fontenc}
\usepackage{lmodern}
\usepackage{amsmath}
\usepackage{amssymb}
\usepackage{mathrsfs}
\usepackage{tikz}
\usepackage{graphicx}
\usepackage{placeins}
\usepackage{listings}
\usepackage{cancel}
\usepackage{hyperref}
\usepackage{xcolor}
\colorlet{punct}{red!60!black}
\definecolor{background}{HTML}{EEEEEE}
\definecolor{delim}{RGB}{20,105,176}
\colorlet{numb}{magenta!60!black}
\lstdefinelanguage{json}{
    basicstyle=\normalfont\ttfamily,
    numbers=left,
    numberstyle=\scriptsize,
    stepnumber=1,
    numbersep=8pt,
    showstringspaces=false,
    breaklines=true,
    inputencoding=utf8,
    extendedchars=true,
    frame=lines,
    backgroundcolor=\color{background},
    literate=
        *{0}{{{\color{numb}0}}}{1}
        {1}{{{\color{numb}1}}}{1}
        {2}{{{\color{numb}2}}}{1}
        {3}{{{\color{numb}3}}}{1}
        {4}{{{\color{numb}4}}}{1}
        {5}{{{\color{numb}5}}}{1}
        {6}{{{\color{numb}6}}}{1}
        {7}{{{\color{numb}7}}}{1}
        {8}{{{\color{numb}8}}}{1}
        {9}{{{\color{numb}9}}}{1}
        {:}{{{\color{punct}{:}}}}{1}
        {,}{{{\color{punct}{,}}}}{1}
        {\{}{{{\color{delim}{\{}}}}{1}
        {\}}{{{\color{delim}{\}}}}}{1}
        {[}{{{\color{delim}{[}}}}{1}
        {]}{{{\color{delim}{]}}}}{1}
        {é}{{\'e}}{1}%
        {è}{{\`e}}{1}%
        {à}{{\`a}}{1}%
        {ç}{{\c{c}}}{1}%
        {œ}{{\oe}}{1}%
        {ù}{{\`u}}{1}%
        {É}{{\'E}}{1}%
        {È}{{\`E}}{1}%
        {À}{{\`A}}{1}%
        {Ç}{{\c{C}}}{1}%
        {Œ}{{\OE}}{1}%
        {Ê}{{\^E}}{1}%
        {ê}{{\^e}}{1}%
        {î}{{\^i}}{1}%
        {ô}{{\^o}}{1}%
        {û}{{\^u}}{1}%
        {ë}{{\¨{e}}}1
        {û}{{\^{u}}}1
        {â}{{\^{a}}}1
        {Â}{{\^{A}}}1
        {Î}{{\^{I}}}1,
}
\newcommand{\deriv}{\mathrm{d}}
\usepackage{array,multirow,makecell}
\usepackage[top=3cm, bottom=3cm, left=3cm,right=3cm]{geometry}
\usepackage{bbold}


\newtheorem{lemma}{Lemme}

\usepackage{fancyhdr}
\pagestyle{fancy}
\fancyhead[L]{M. \bsc{Augé} et M. \bsc{Roux}}
\fancyhead[C]{}
\fancyhead[R]{DistribEauPti - Rapport}
\renewcommand{\headrulewidth}{1pt}
\fancyfoot[C]{\thepage}

\newcolumntype{C}[1]{>{\centering\arraybackslash }b{#1}}
\setcounter{MaxMatrixCols}{20}
\renewcommand{\footrulewidth}{1pt}

\title{DistribEauPti \\ Rapport}
\date{\today}
\author{\bsc{Augé} Marc-Antoine et \bsc{Roux} Matthieu}

\begin{document}
\thispagestyle{empty}
\begin{center}

    \begin{figure}[!htb]
        \begin{center}
            \includegraphics[width=4cm]{logoPonts.png}
        \end{center}
    \end{figure}
    
    \vspace{0.5cm}
    
    {\large{\bf École des Ponts ParisTech}}
    
    \vspace{0.2cm}
    
    {\large{\bf Mars 2018 - Avril 2018}}
    
    \vspace{1.5cm}
    
    \large{ \bf Cours Optimisation et Contrôle}\\
    \vspace{0.2cm}
    {\Large{\bf DistribEauPti - Projet sur les réseaux de distribution d'eau}}
    
    \vspace{1cm}
    
    \large{Marc-Antoine Augé et Matthieu Roux}

\end{center}
\newpage

\tableofcontents
 
\section{Séance 1 - Définition de l'oracle}
    \subsection{Calculs différentiels}
        On pose \[ F(q) = \frac{1}{3}<q, r \circ q \circ |q|> + <p_r, A_rq> \] et \[q(q_c) = q^{(0)} + Bq_c \]
        On cherche à calculer $\nabla F(q(q_c))$ et $H F(q(q_c))$ le Hessien.\\
        Remarquons tout d'abord que les matrices sont à coefficients réels donc transposition et adjonction sont deux opérations identiques.\\
        Commencons par $\nabla F(q)$ en écrivant le produit de Hadamard terme à terme et le produit scalaire sous forme de somme :
        \[ F(q) = \frac{1}{3}\sum_{i = 1}^n q_i^2.r_i.|q_i| + <p_r, A_rq>\]
        On note alors $\epsilon_i$ le signe de $q_i$ :
        \[ F(q) = \frac{1}{3}\sum_{i = 1}^n \epsilon_i.r_i.q_i^3 + <p_r, A_rq>\]
        D'où immédiatement, étant donné que le gradient du second terme est $A_r^T.p_r$ :
        \[\nabla F(q) = (\epsilon_i.r_i.q_i^2)_{1 \leq i \leq n} + A_r^T.p_r\]
        On peut le réécrire sans la notation $\epsilon_i$ et en utilisant un produit de Hadamard : 
        \[ \boxed{\nabla F(q) = (r_i.q_i.|q_i|)_{1 \leq i \leq n} + A_r^T.p_r
            = r\circ q \circ |q| + A_r^T.p_r}\]

        Puis par composition, comme $q(q_c) = q^{(0)} + q_c$, on a, en notant par abus de notation : $F(q_c) = F(q(q_c))$ :
        \[ \boxed{\nabla F(q_c) = B^T\nabla F(q) = B^T(r\circ q \circ |q| + A_r^T.p_r)  
        }\]

        Pour calculer le Hessien, on a tout d'abord, en notant $\text{diag}(X)$ la matrice diagonale possédant sur sa diagonale les coefficients de $X$ :
        \[H F(q) = \text{diag}(r\circ(q + |q|))\]
        Par composition, étant donné que le $H(q(q_c)) = 0$ :
        \[ \boxed{H F(q_c) = H F(q(q_c)) = B^T.\text{diag}(r\circ(q + |q|).B}
        \] 

    \paragraph{}
\section{}

\bibliographystyle{abbrv}
\bibliography{biblio}
\end{document}
